% *****************************************************************************
% **************************** Preámbulo ********************************

\usepackage[left=2.5cm,right=2.5cm,top=2.5cm,bottom=2.5cm]{geometry} %Definición de márgenes
%\usepackage{titling}  %para configurar título en pagina de tapa
\usepackage{authblk}  %para configurar nombres y afiliaciones de autores
\usepackage{parskip}  %para separar parrafos con línea en blanco
\parskip=6pt          % se adopta separación de 6pt entre parrafos
%este paquete además setea a cero la indentación

\usepackage{setspace} %separacion entre líneas de texto (\doublespacing or \onehalfspacing)
\onehalfspacing       %para reportes adoptamos espaciado 1y1/2 entre líneas

\usepackage[spanish,es-tabla,es-lcroman]{babel}

%	IDIOMAS - RECONOCIMIENTO DE CARACTERES EN ESPAÑOL
\usepackage[utf8]{inputenc} %reconoce entrada caracteres internacionales - iso-8889-1
\usepackage[T1]{fontenc} %para salida mejorada caracteres internacionales
\usepackage{lmodern} %aporta fuentes faltantes en T1

%	PAQUETES PARA FORMATO DEL DOCUMENTO
\usepackage{graphicx} %para incluir figuras
\usepackage{float}    %para tablas y figuras flotantes (que entran en una pagina)
\usepackage{subcaption} %para insertar multiple figuras

\usepackage[hidelinks]{hyperref} %genera hipervínculos a titulos, figuras, ecuaciones
\hypersetup{         %metadata para PDF
	colorlinks=false, %set true if you want colored links
	linktoc=all,      %set to all if you want both sections and subsections linked
	%linkcolor=blue,  %choose some color if you want links to stand out
} 

\usepackage{titlesec}  % para eliminar la palabra "Capítulo" de títulos
\usepackage{framed} % para enmarcar paragrafos
\usepackage{breakcites} %para cortar referencias muy largas

%	PAQUETES PARA ECUACIONES Y TABLAS
\usepackage{amsmath} % paquete para escribir ecuaciones matemáticas 
\usepackage{amssymb} % paquete de simbolos matemáticos
\usepackage{mathtools} % paquete de mejoras para ecuaciones (amsmath)  

\usepackage[fleqn]{empheq} % subecuaciones enmarcadas
\usepackage{nccmath} % para alinear ecuaciones aisladas a la izquierda

\usepackage{booktabs} % para lineas de separación en tablas
\usepackage{siunitx}  % numeros en notación científica y unidades

\usepackage{enumerate} % para hacer listas enumeradas

\usepackage{ifthen}    %paquete para comparaciones lógicas if-then-else

\usepackage{csquotes} % necesario para evitar conflicto con biblatex
% Carga de estilo APA para la tesis - No modificar
\usepackage[backend=biber,style=apa,sortcites,natbib=true]{biblatex}
\DeclareLanguageMapping{spanish}{spanish-apa}  % APA en español

% ESPAÑOL - Para que se coloque "y" y no "&" como delimitador de autores con estilo APA.

\DeclareDelimFormat*{finalnamedelim}
{\ifnum\value{liststop}>2 \finalandcomma\fi\addspace\bibstring{and}\space}

% the bibliography also needs another conditional, so we can't wrap
% everything up with just the two lines above
\DeclareDelimFormat[bib,biblist]{finalnamedelim}{%
	\ifthenelse{\value{listcount}>\maxprtauth}
	{}
	{\ifthenelse{\value{liststop}>2}
		{\finalandcomma\addspace\bibstring{and}\space}
		{\addspace\bibstring{and}\space}}}

% Esta sección es para que se coloque "y" y no "&" como delimitador de autores en Referencias.
\DeclareDelimFormat*{finalnamedelim:apa:family-given}{%
	\ifthenelse{\value{listcount}>\maxprtauth}
	{}
	{\finalandcomma\addspace\bibstring{and}\space}}

% ESPAÑOL - Para que escriba "et al." en español.
\DefineBibliographyStrings{spanish}{%
	andothers = {et al.},
}
\addbibresource{Bibliografia/bibliografia.bib} % Ubicación de bibliografia.bib a cargar, no omita la extensión .bib del nombre de archivo.